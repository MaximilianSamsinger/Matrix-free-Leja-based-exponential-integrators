\documentclass{scrartcl}
\usepackage[utf8]{inputenc}
\usepackage[ngerman,english]{babel}
\usepackage{blindtext}
\usepackage{color}
\usepackage{graphicx}
\usepackage{grffile}
\usepackage{subcaption}
\usepackage{upgreek}
\usepackage{amsmath,amssymb,amsthm,mathtools}
\usepackage{multirow}
\usepackage{float}
\usepackage{hyperref}

\newcommand{\argmax}{\operatornamewithlimits{argmax}}
\newcommand{\defneq}{\mathrel{\mathop:}=}
\newcommand{\eqdefn}{=\mathrel{\mathop:}}

\DeclarePairedDelimiter\abs{\lvert}{\rvert}%
\DeclarePairedDelimiter\norm{\lVert}{\rVert}%

\begin{document}
\section{Analysis of the power iteration}
\begin{figure}[t]
	\newcommand{\boundary}{periodic}
	\centering
	\includegraphics[width=1.\columnwidth]{{../figures/Spectrum/\boundary}.pdf}
	\caption{The spectrum of $A$. For this visualization we assume \boundary\ boundary conditions. For Dirichlet boundary conditions all eigenvalues are negative real numbers.}
\end{figure}

We analyse the rate of convergence of the power method for the linear advection-diffusion equation with periodic boundary conditions. Let 
\[F(u) = a\partial_xu + b\partial_{xx} u \]
be the right hand side of the differential equation. Furthermore let $u\in L^2([0,1])$. The differential equation
\[
	
\]


The eigenvalues of discretized one-dimensional Laplace operator $A_{Dif}\in\mathbb{R}^{N\times N}$ on the interval $[0,1]$ with periodic boundary conditions are given by

\[
\lambda_j =
\begin{cases*}
-\frac{4}{h^2} \sin^2\left(\frac{\pi (j-1)}{2(N+1)}\right),\quad\text{if j is odd}\\
-\frac{4}{h^2} \sin^2\left(\frac{\pi j}{2(N+1)}\right),\quad\text{if j is even}
\end{cases*} 
,\quad j=1,\dots, N.
\]
We investigate the rate of convergence for the power method given $A_{Dif}$ and an initial vector $v$. Consider $v=\frac{1}{N}\sum_{j=1}^N v_{j}$, where each $v_j$ is the normalized eigenvector corresponding to the eigenvalue $\lambda_j$.
After $n$ power iteration we underestimate the absolutely largest eigenvalue $\lambda_N$ by a factor of
\begin{align*}
\frac{\norm{A_{Dif}^{n-1}v}_2}{\norm{A_{Dif}^nv}_2}\abs{\lambda_N} = 
\sqrt{\frac{\sum_{j=1}^{N}{\lambda_j^{2(n-1)}}}{\sum_{j=1}^{N}{\lambda_j^{2n}}}}\abs{\lambda_N} = 
\sqrt{\frac{\sum_{j=1}^{N}\sin^{4(n-1)}
		\left(\frac{\pi j}{2(N+1)}\right)}{\sum_{j=1}^{N}\sin^{4n}
		\left(\frac{\pi j}{2(N+1)}\right)}}\sin^2\left(\frac{\pi N}{2(N+1)}\right).
\end{align*}
The first equality holds since $A_{Dif}$ is symmetric. Therefore all eigenvectors are orthogonal. In order to continue our analysis and get some asymptotic bounds we interpret the sum of sine functions as an integral approximated by the trapezoidal rule. We use the nodes $j/(N+1)$ for $j=0,\dots,{N+1}$.
\[
\int_{0}^{1}{\sin^{4n}\left(\frac{\pi x}{2}\right)} =
\frac{1}{(N+1)}\left(2\sum_{j=1}^{N}\sin^{4n}\left(\frac{\pi j}{(N+1)}\right) + \frac{1}{2}\right)
+ \mathcal{O}\left(\frac{1}{12(N+1)^2}\right)
\]
Note that the error of the approximation is strictly positive since the second derivative  math.stackexchange\footnote{\url{https://math.stackexchange.com/questions/50447/integration-of-powers-of-the-sin-x}} we can blissfully accept the identity
\[I_n \defneq \int_{0}^{1}{\sin^{4n}\left(\frac{\pi x}{2}\right)} = \frac{\Gamma(2n+0.5)}{\sqrt{\pi}\ \Gamma(2n+1)}. \]
In order to simplify our calculations we take the limit of $N$
\[ 
\frac{\norm{A_{Dif}^{n-1}v}}{\norm{A_{Dif}^{n}v}}\abs{\lambda_N} \xrightarrow[]{N\to\infty}
\sqrt{\frac{I_n}{I_{n-1}}}, 
\]
where

\begin{align*}
\frac{I_n}{I_{n-1}} &= 
\frac{\Gamma(2n - 1)\Gamma(2n + 0.5)}{\Gamma(2n + 1)\Gamma(2n - 1.5)} \\&=   
\frac{(2n - 2)!}{(2n)!}
\frac{\Gamma(2n + 0.5)}{\Gamma(2n - 1.5)}
\frac{\Gamma(2n)}{\Gamma(2n)}
\frac{\Gamma(2n-2)}{\Gamma(2n-2)} \\&=
\frac{1}{2n(2n-1)} 
\frac{2^{1-4n}\sqrt\pi}{2^{5-4n}\sqrt\pi}
\frac{\Gamma(4n)}{\Gamma(4n-4)}
\frac{\Gamma(2n-2)}{\Gamma(2n)} \\&=
\frac{1}{32n(2n-1)} 
\frac{(4n-1)!}{(4n-5)!}
\frac{(2n-3)!}{(2n-1)!} \\&=
\frac{(4n-1)(4n-2)(4n-3)(4n-4)}{32n(2n-1)^2(2n-2)} \\&=
\frac{(4n-1)(4n-3)}{8n(2n-1)} \\&=
\frac{4n-1}{4n}\frac{4n-3}{4n-2} \\&=
\left(1-\frac{1}{4n}\right)\left(1-\frac{1}{4n-2}\right)
%%\\&=1 - \frac{1}{2n}\frac{8n-3}{8n-4} 
\end{align*}
For the third equality we applied the duplication formula for the gamma function. All in all we underestimate the absolutely largest eigenvalue $\lambda_N$ by a factor of 
\[
\lim_{N\to\infty}\frac{\norm{A_{Dif}^{n-1}v}}{\norm{A_{Dif}^{n}v}}\abs{\lambda_N} =
\sqrt{\left(1-\frac{1}{4n}\right)\left(1-\frac{1}{4n-2}\right)} \approx
1-\frac{1}{4n-1}
\]
at the limit $N\to\infty$. 
\end{document}